\section{Méthodologie et théorie en conception}
\label{sec:Méthodologies et théorie de la conception}
Dans cette section nous allons observer les méthodes de conception.
D'une part elles nous seront nécessaires dans l'analyse critique de l'\gls{ACV} comme dans le questionnement des \gls{UF}, d'autre part nous pourrons les employer pour traiter de l'\emph{éco-conception}.
Nous soulignerons la proximité et les distinctions potentielles entre évaluation, conception, éco-conception.

Nous traiterons de la conception avec d'une part une orientation pratique avec les ouvrages de \cite{philippe_taillard_demarche_2010,tassinari_pratique_1997}~\citeauthor{philippe_taillard_demarche_2010,tassinari_pratique_1997}.
D'autre part nous travaillerons une orientation théorique.
Nous nous baserons notamment pour cette seconde sur l'ouvrage d'épistémologie de \citeauthor{micaelli_artificialisme:_2003}, \citetitle{micaelli_artificialisme:_2003}~\cite{micaelli_artificialisme:_2003}.

Nous conclurons que plus des proximités c'est une imbrication des processus d'évaluation et de conception qui a lieu, l'éco-conception n'étant qu'une orientation particulière de l'évaluation pour la conception.
\subsection{Aux origines de la conception}
Pour approcher le \blockcquote{yannou_preconception_2001}{domaine des "design theories and methodologies"},
nous nous référerons tout d'abord à \citetitle[2.3.4]{yannou_preconception_2001} de \citeauthor{yannou_preconception_2001}, traitant de la science de la conception.
D'après \citeauthor{yannou_preconception_2001},
\blockcquote{yannou_preconception_2001}{
ce domaine est né de la nécessité de conceptualiser la conception qui était quelque peu vue avant les années 1960 comme l’activité d’un « homme de l’art ».
Perrin explique fort bien dans [Perrin 2001b] comment dans les années 1960 à 1970, la conception a commencé à être théorisée pour devenir le domaine des design theories et des design methodologies. Ceci a commencé en Angleterre puis aux Etats-Unis et en Allemagne.
Quelques professeurs de conception ont théorisé leur savoir dans des ouvrages de référence, ce que nous appellerons les approches générales.
}

Notre opinion sur la raison de cette perception duale est la suivante.
La dichotomie science - art dans ce domaine tient au fait que la conception porte une activité de détermination, de priorisation des besoins et fonctions auxquels le praticien tente d'offrir une nouvelle réponse par l'entremise d'un artefact.
La subjectivité sera de nouveau employée pour \emph{juger} de la qualité des différentes réponses (solutions techniques) élaborées.
Sans pouvoir faire clairement la distinction entre domaine factuel (science) et la composante axiologique (valeurs conduisant à l'action), la conception était catégorisée, à juste titre, d'art. 

Sans quitter la volonté d'analyse scientifique du processus de conception nous étendons l'historique.
Nous repousserons donc volontiers en amont l'histoire de la conception en acceptant des origines situées dans "l'art", tout d'abord vers 1920 avec \citeauthor{bayazit_investigating_2004}.
\blockcquote[traduction]{bayazit_investigating_2004}{
De nombreux auteurs ont souligné De Stijl au début des années 1920 comme un exemple de la volonté de "rendre scientifique" la conception.
Les racines de la recherche en conception dans de nombreuses disciplines depuis les années 1920 se trouvent dans le Bauhaus, qui a été créé comme la fondation méthodologique pour l'enseignement de la conception.}

%Many writers 5 have pointed to De Stijl in the early 1920s as an example of the desire to “scientize” design. The roots of design research in many disciplines since the 1920s are found within the Bauhaus, which was established as the methodological foundation for design education.}


Puis avec \citeauthor{micaelli_artificialisme:_2003} et traitant de la théorie de l'artificialisme, nous changerons d'échelle historique (\textsc{Empédocle} quatrième siècle av. J-C. ; \textsc{Lucrèce} premier siècle av. J-C.)).
\blockcquote[chapitre 3, p.48 3.3 Brève histoire de l'Artificialisme]{micaelli_artificialisme:_2003}{
Le philosophe français Clément \textsc{Rosset} regroupe sous le mot d'Artificialisme l'ensemble des doctrines opposées au Naturalisme.
Celles-ci auraient été énoncées par des auteurs antiques comme \textsc{Empédocle} ou \textsc{Lucrèce} (Rosset, 1995:127) pour ce qui relève de la Physique, ou encore par Thomas \textsc{Hobbes} (1588-1679) pour ce qui concerne l'organisation de la société (\textsc{Hobbes}, 1999:81).
}
Nous voyons donc que la recherche méthodologique de production et rationalisation d'un artefact est très distante de notre ère.
Déjà, l'introduction du jugement et de la subjectivité et perçu comme une frontière à la science.
La reconnaissance d'une science \textit{sur} une pratique requérant l'application du jugement (la subjectivité) est refusé de longue date.

\subsection{Éléments théoriques}
\label{subsection:Éléments théoriques}
Posons tout d'abord des éléments de vocabulaire.
\keybox{
\blockcquote[chapitre 3, p.46 3.2 Artefact, une définition]{micaelli_artificialisme:_2003}{
Est artefact toute entité, tangible ou non, conçue en vue de répondre à des besoins, `bien que ce ne soit pas toujours [\ldots] avec une claire vision anticipatrice' (Simon, 1991B:7), `synthétisée' par l'Homme (Simon, 1991B:7), puis rationalisée, c'est à dire reconçue.
}
}
L'\gls{ACV} est donc un artefact visant la confrontation d'autres artefacts (les alternatives de produits ou services) répondant à un ou des besoins similaires, \emph{commun} (l'\gls{UF}).


\blockcquote[p.124-128, emphase personnelle]{micaelli_artificialisme:_2003}{
Le modèle de Pahl et Beitz décrit \textbf{la conception comme un processus de formulation et de résolution de problèmes} multiétapes que l'on peut nommer par les termes suivants~:
	\begin{itemize}
\item initialitaiton du projet de conception [\ldots];
\item réflexion préliminaire ou étude préliminaire [\ldots];
\item étude d'avant-projet ([\ldots] "conceptual design");
\item étude et réalisation d'une solution globale[\ldots];
\item étude d'exécution ou de réalisation de solution détaillées[\ldots];
\item fin du projet de conception
	\end{itemize}
}
La description de la conception de \citeauthor{micaelli_artificialisme:_2003} reprendra le vocabulaire de l'\gls{AFAV} et notamment d'APTE\footnote{Méthode de conception du cabinet de conseil éponyme (CF leur \href{http://methode-apte.com/}{site}).}.

Mais un autre modèle retient notre attention.
Dans leur chapitre 5 \citeauthor{micaelli_artificialisme:_2003} situent la prise de décision comme une partie du processus de conception~\cite[p.81]{micaelli_artificialisme:_2003}.
Ils ajoutent que 
\blockcquote[p.82]{micaelli_artificialisme:_2003}{
plus encore le "modèle canonique" (Le moigne, 1990a:17) de quatre étapes faisant se succéder l'\textbf{intelligence du problème}, la \textbf{recherche de solution} candidates, le \textbf{choix} d'une solution parmi les candidates et le \textbf{bilan} (Simon 1980:35) s'applique également au processus de conception.
}
Nous ajouterons que ce modèle canonique s'applique également à l'\gls{ACV}.
La formulation en quatre étapes nous incite d'ailleurs à faire le parallèle avec la méthodologie de l'\gls{ACV} décrite par l'\gls{ISO}.
\begin{itemize}
\item L'intelligence du problème concentre les questions de formulation du but, de l'unité fonctionnelle et du périmètre attenant.
\item La recherche de solutions comprend la considération des systèmes alternatifs et la réalisation de leurs inventaires.
\item Le choix comprend la caractérisation, l'évaluation.
\item Le bilan est l'interprétation finale, la préconisation d'une alternative préférée.
\end{itemize}


Poursuivons l'analyse des similitudes.
\blockcquote[p.82]{micaelli_artificialisme:_2003}{
Les processus de décision et de conception %[\ldots]
partagent tous deux quelques traits remarquables.
Ils sont \textbf{itératifs} car toutes les étapes bouclent entres elles~: le décideur/concepteur ne pouvant anticiper complètement ce qu'il fera aux étapes suivantes.
Ils sont globalement \textbf{convergents} et aboutissent tous deux à une solution satisfaisante en temps fini.
Ils sont \textbf{hiérarchiques et partiellement irréversibles}~: les conjectures et le contenu d'une étape aval dépend du résultat de l'étape amont.
Enfin, à mesure que le décideur/concepteur avance dans le processus de décision, la résolution de problème prend progressivement le pas sur l'activité de formulation de problème.
%Lors des étapes amont, notamment lors de la phase d'intelligence, il lui faut cerner le problème, extraire de l'environnement externe des données qui paraissent pertinentes pour une stratégie donnée, puis ensuite focaliser sont attention sur l'imagination, la réalisation, le choix de solutions.
}
L'\gls{ACV} en va de même en stabilisant progressivement objectifs, \gls{UF} et périmètres pour traiter des scénarios, des inventaires et de la sélection / identification de l'alternative préférée.

\citeauthor{arnoux_modeliser_2013} dans \citetitle{arnoux_modeliser_2013} précise le développement des méthodes de créativité.
\blockcquote{arnoux_modeliser_2013}{
%Dans ce sillage, de nombreuses méthodes et techniques de créativité ont émergé à partir des années 1950.
Dans le management de la créativité, les plus connues sont les techniques de Creative Problem Solving (CPS) et de brainstorming développées par Osborn (1957).
Leur principe de base est de générer des idées créatives en sessions de groupes dans lesquelles plusieurs règles permettent de suspendre le jugement.[\ldots]
%Néanmoins, malgré leur succès dans le monde de l’industrie, plusieurs études démontrent que les brainstormings conduisent aussi à la génération de troubles d’attention durant les sessions de groupe (Mulligan \& Hartman, 1996), à la génération d’anxiété sociale (Camacho \& Paulus, 1995), ainsi qu’à des effets négatifs dûs à l’expertise (Collaros \& Anderson, 1969).
Dans cette lignée, des techniques de CPS ont également été développées pour accélérer la résolution de problèmes en utilisant des connaissances existantes pour résoudre des contradictions techniques. [\ldots]
%Ainsi, les travaux sur la créativité de groupe proposent un large champ d’outils et de méthodes visant à se dégager des règles de conception dominantes, et à générer des idées créatives dans les entreprises.
%Pour un état de l’art de ces techniques voir e.g. McFadzean, E. 1998.
%The Creativity Continuum: Towards a Classification of Creative Problem Solving Techniques.
%Creativity and Innovation Management, 7(3): 131-139. 211 existantes.
%Cependant, l’intérêt de ces séances est centré sur la fédération d’idées nouvelles à travers un groupe plutôt que sur l’originalité des propositions.
%Dans la Figure 38, ces méthodes sont placées à droite sur l’axe « niveau d’adhésion requis ».
}
Nous ne détaillerons pas les questions de créativité dont la classification est faite par \citeauthor{mcfadzean_creativity_1998}~\cite{mcfadzean_creativity_1998}, comme le souligne \citeauthor{arnoux_modeliser_2013}.
En effet, les solutions techniques présentent dans diverses disciplines n'ont pas rendu nécessaire l'emploi de ce type de techniques dans notre cas d'étude et la description de ce vaste domaine alourdirait inutilement le présent document.
Prenons toutefois le temps de la remarque suivante.

La créativité dans l'activité d'\gls{ACV} repose dans la considération des alternatives.
Nous pouvons toutefois remarquer que si \citeauthor{micaelli_artificialisme:_2003} décrivent la conception, l'artificialimse, comme la créativité suivi de la rationalisation\cite[p.51]{micaelli_artificialisme:_2003}, alors nos travaux sur l'\gls{ACV} portent majoritairement sur la seconde.
Il ne faudra toutefois pas oublier qu'au delà de cette thèse l'ACV doit faire appel aux techniques de créativité.

Puisqu'il s'agit de la rationalisation d'un système, il convient de discuter de performance.
Celle-ci peut être perçue comme un rendement de valeurs entre la modification des états de l'environnement par le système (valeurs globales résultantes) pour l'obtention de la finalité (valeur(s) recherchée(s)).
Toutefois, \blockcquote{yannou_preconception_2001}{
la notion de valeur est une notion bien mal partagée.
Tout le monde s’accorde à dire qu’il faut augmenter chacun des maillons de la chaîne de la valeur (voir Porter [Porter 1986]).
Mais cette notion reste très polysémique (voir Perrin [Perrin 2001a], Chazelet et Lhote [Chazelet et al 2001], et Ben Ahmed [Ben Ahmed 2001]).
En conséquence, peu d’indicateurs réellement opérationnels permettent de la chiffrer durant le projet de conception (voir aussi le chapitre 5.5 en page 115) pour concevoir le meilleur produit.
}

Ce que \citeauthor{yannou_preconception_2001} souligne avec moult auteurs dans le passage précédent comme polysémie, nous le percevons comme le caractère intrinsèque de la valeur dans un environnement 'poly-sujets', à multiple subjectivités en milieu incertain.
Il ne peut donc pas y avoir \emph{par nature} de meilleur produit, juste un produit préféré par un ou des sujets

\citeauthor{yannou_preconception_2001} développe, dans son Habilitation à Diriger la Recherche, trois thématiques qu'il semble à propos de mentionner ici~:
\blockcquote{yannou_preconception_2001}{
\begin{itemize}
\item l’évaluation multi-critère de solutions,
\item la représentation de l’imprécision en préconception,
\item le lien entre paramètres de conception, performances et besoins.
\end{itemize}}
%A part ces approches générales, ce domaine de recherche est riche en thématiques abordées, nous choisirons d’en aborder quelques-unes ici qui nous semblent importantes en conceptual design :
%- l’évaluation multi-critère de solutions,
%- la représentation de l’imprécision en préconception,
%- le lien entre paramètres de conception, performances et besoins.
%En France, très peu de chercheurs nous semblent travailler spécifiquement dans
%ce domaine. Il s’agit d’un domaine majeur de notre thème de recherche et sur lequel nous allons nous recentrer particulièrement dans le futur.}

Le lecteur comprendra (une nouvelle fois s'il en était besoin) à l'évocation de ces thématiques la proximité disciplinaire de l'évaluation environnementale, de la conception et de la décision multi-critère.
L'évaluation multi-critère est traitée spécifiquement en section~\ref{sec:ADMC}.
%La représentation de l'imprécision sera discuté en section~\ref{descriptionsemantique} (modèles et descriptions).
La question des liens entre 'conception', performances et besoins est traitée dans la question de la multidimensionnalité~\ref{chap:Jugements et Multi-dimensionnalité}.

Nous souhaitons approfondir ces axes avec l'approche artificialiste de l'évaluation.
\blockcquote[p.7]{micaelli_artificialisme:_2003}{Nous avons vu que la conception est une activité évaluative.
\ldots

Celui qui mène l'évaluation, ou évaluateur, doit concevoir un système un véritable artefact, appelé système d'évaluation de performance.
Celui-ci permet de saisir la situation d'évaluation dans laquelle notre évaluateur se place, c'est à dire pourquoi l'évaluation est menée, pour qui, etc.
Ce système permet d'appréhender la procédure d'évaluation suivie, puis les évaluants, c'est-à-dire les briques de base (les objectifs, les indicateurs, les critères, etc.), utilisés.
%\ldots
}
Ceci est synthétisé par le titre de section de l'ouvrage de \citeauthor{micaelli_artificialisme:_2003} attenant~:\blockcquote[9.1.1]{micaelli_artificialisme:_2003}{Évaluer, c'est concevoir un système d'évaluation}.

\begin{table}[htbp]
	\centering
		\begin{tabular}{p{2.5cm}|p{2.5cm}|p{5cm}}
			Niveau & Description & Question posées \\
			\hline
			pragmatique & situation d'évaluation & Qui évalue ? Pour qui ? Pour contrôler quelle transaction ? Pour contrôler quel système cible ? etc. \\
			procédural & procédure d'évaluation & Comment ? En partant de quelle étape ? En enchaînant quelles étapes ? En allant jusqu'où ? \\
			objet & système d'évaluants & Avec quels évaluants ? Qui contribuent à quels évaluants ? etc. \\
			informationnel & système d'information & Quel est le système d'information support des trois niveaux précédents ? etc.
		\end{tabular}
	\caption{\blockcquote[p.163]{micaelli_artificialisme:_2003}{
	Table 9.1 Architecture du système d'évaluation.}}
	\label{tab:ArchiSystEvaluation}
\end{table}

\figbox{
Voyons donc de quoi est fait un tel système avec la table~\ref{tab:ArchiSystEvaluation}.

%Nous ne retiendrons que trois niveau en relation avec notre thématique d'étude.
Le premier niveau questionne dans le cas de l'\gls{ACV} le décideur, l'analyste, le but et le périmètre.
Le second serait le tracé des ISO ou de l'ILCD.
Le troisième si nous saisissons les auteurs en réfère à la partie objective, l'observation pure servant de base à la décision rationnelle.
Le dernier niveau nous interroge sur la structure des données d'\gls{ACV}.
}


L'activité de \citeauthor{micaelli_artificialisme:_2003} vise au développement d'
\blockcquote[p.8-9]{micaelli_artificialisme:_2003}{
\textbf{une science de la conception},
le projet esquissé par SIMON d'une science de la conception dont dériverait la science du système de production, qui n'est qu'un artefact particulier.
}
Or l'ACV vise précisément `l'interaction avec' et `la modification' de ce système de production.
Il parait alors tout à fait logique d'étudier l'artificialisme tant pour (i) ce qu'il nous apprend des systèmes de transformation de la matière que nous observons, que pour (ii) la conception du système d'évaluation que nous souhaitons appliquer à ces systèmes.


Pour développer cette science de la conception \citeauthor{micaelli_artificialisme:_2003} confrontent Rationalisme et Traditionalisme avant de positionner l'Artificialisme.

Selon l'
\blockcquote[p.25]{micaelli_artificialisme:_2003}{
\textbf{approche rationaliste de la conception}
[\ldots]
Il n'y a pas d'effet d'échelle qui justifierait de procédures de calcul différentes selon le degré de complication de l'artefact concerné.
}
Cette vision est donc conforme à la mise en place d'une méthodologie générale de l'\gls{ACV} quelque soit le cadre décisionnel et le système visé.

Discutant de l’archétype rationaliste, \citeauthor{micaelli_artificialisme:_2003} précisent que
\blockcquote[2.1.2 p.25]{micaelli_artificialisme:_2003}{
%\textbf{Économie rationaliste cognitiviste}
\ldots
La concrétisation du calcul, les modalités effectives de l'action, ne l'intéresse pas.
Imaginer les conséquences d'un choix suffit au Rationaliste pour décrire l'ensemble d'une action.
[\ldots]
Il manifeste soit un attentisme permanent, soit un constructivisme radical.
}
Les deux tendances consistent (i) en un retardement constant de l'action pour accroître la qualification des choix (réductions des incertitudes) (ii) l'avancée vers un idéal, à marche forcée dans l'action.

\blockcquote[p.27]{micaelli_artificialisme:_2003}{
Le constructivisme radical [\ldots] procède en deux temps.
Il admet qu'il existe un esprit individuel capable d'imaginer une solution idéale, cohérente, intrinsèquement bonne, quelle que soit la taille du problème posé.
L'action idéale consiste ensuite à appliquer cette solution, coûte que coûte [\ldots]
Si résistance il y a dans la mise en œuvre, ce ne peut qu'être dû qu'à des facteurs irrationnels, la résistance au changement, l'inertie cognitive, l'irrationalité profonde, des destinataires de ces changements.}

Cette combinaison de tendances s'observe dans les pratiques d'éco-conception et d'ACV (hors \textit{greenwashing} volontaire).
D'une part la tendances attentiste pour ce qui est décrit comme des approches 'comptables' qui ne conduirait aucune action, voir qui se considérerait objective (cas décrit en ILCD)~\cite[5.3.7 Situation C, accouting]{european_commission_ilcd_2010}\footnote{
\blockcquote{european_commission_ilcd_2010}{
Purely descriptive accounting / documentation of the analysed system
}.
}.
D'autre par les études à posteriori, justificatrices de la \emph{bonne solution} appliquée\footnote{Il semble délicat de ne pas tomber dans la catégorie 'greenwhasing' avec ces études à postériori des choix}.

Le traditionalisme vise la conservation des artefacts antérieurs sur le principe qu'aucun individu ne peut égaler l'historique de sélection desdits artefacts dans sa production d'un artefact nouveau surpassant les anciens.
Il s'agit des mécanismes de conservations.
\blockcquote[Traditionalisme p.28]{micaelli_artificialisme:_2003}{
Dès lors, tout artefact n'est pas le fruit d'un choix, d'un calcul préalable, détaché, décontextualisé, comme le croit le Rationaliste.
Il résulte de la mise en œuvre intuitive de pratiques conformes à un ordre éternel (Traditionaliste fixiste) ou résultant de "`l'observance [\ldots] de pratiques [routinières], qui ont prévalu parce qu'elles réussissaient"' (Von Hayek, 1980:20) (Traditionaliste évolutif).
}
Nous pourrions y voir la tendance au rejet des approches complexes pour le maintien des systèmes de résolution par discrétisation des problèmes.
Rappelons que certains auteurs de la discipline proposaient l'abandon pur et simple de l'ACV parce que la complexité à atteindre était jugé hors de portée~\cite{bare_life_1999}.

Nous approchons d'une définition de l'artificialisme, donc de la conception, par négatif et complémentarité du traditionalisme et du rationalisme. 
\blockcquote[Traditionalisme p.35]{micaelli_artificialisme:_2003}{
Pour l'Artificialiste, supposer qu'existe un \textit{Cosmos} (Traditionalisme) ou une nature (Rationalisme) qui prédéterminerait toute action constitue un obstacle épistémologique majeur à la compréhension de ce qu'est la production, la conception ou l'innovation.
}
Lorsque l'artificiel est défini, une notion similaire apparaît, avec un résonance particulière à notre problématique de sphères (\gls{technosphere} \gls{ecosphere} et \gls{biosphere}). 
L'opposition entre artificiel et naturel reproduit la dichotomie technosphère - biosphère propre à l'\gls{ACV}.
\blockcquote[chapitre 3, p.39 3.1 Artificialisme naïf]{micaelli_artificialisme:_2003}{
Nous regroupons sous le terme d'Artificialisme naïf deux idées ou préjugés.
La première est que le naturel s'oppose terme à terme à l'artificiel.
La seconde est que le concepteur est un esprit unique, omniscient et omnipotent.
Bien sûr l'Artificialisme défendu dans cet essai s'oppose à celles-ci.
}
Ces tendances semble encore présentes jusqu'en ACV dans la focalisation sur la transformation de l'espace 'naturel' (eco ou bio-sphère) ou lorsque sont discutés des technologies "plus impactantes que"\ldots

Si l'\gls{ACV}, comme l'artificialisme, repose sur des valeurs, elle ne les explique pas.
\blockcquote[chapitre 3, p.51 3.4 Propositions clefs de l'Artificialisme]{micaelli_artificialisme:_2003}{
L'Artificialisme est partiel.
Il ne recouvre pas tout le champ de l'action humaine.
Il ne permet pas d'en expliquer la dimension axiologique, c'est-à-dire comment émergent, se confrontent, les valeurs, les principes moraux ou éthiques, les finalités, les idéologies, etc., qui en donnent un sens ultime.
% Ces questions doivent être traitées par des doctrines plus larges, qui intègrent explicitement la questionde la formation des valeurs, par exemples les praxéologies institutionnaliste (Commons, 1989), weberienne (Weber, 1972), parsonienne (Parsons, Shils, 1967), joasienne (Joas, 1999), etc.
}

\citeauthor{micaelli_artificialisme:_2003} posent une classification des artefacts créés avec intention (classe A)
\footnote{Par opposition aux artefacts de classe N, produit de l'activité humaine mais hors dessein et sans être modifiable de façon délibéré \cite[p.55 Chapitre 4]{micaelli_artificialisme:_2003}}.
Cette classification est reprise table~\ref{tab:artefactsdeclasseA}.
\begin{table}
	\centering
	\begin{tabular}{l|p{4.5cm}|p{4.5cm}}
		Artefact & Fonction & Exemples \\
		\hline
		final & satisfaire un besoin & tous les artefacts \\
		démonstrateur & valider des finalités & artefacts alternatifs \\
		interopératoire & opérer avec un artefact final, pour rendre son fonctionnement effectif & standard, normes, artefacts complémentaires \\
		fabricatoire & fabriquer un artefact & équipement, outillage, bâtiment \\
		intermédiare & aider à la conception d'un artefact & cahier des charges, schéma, dessin, maquette \\
		cognitif & aide aux fonctions et processus cognitifs & moyen mnémotechnique, langage \\
		immersif & plonger l'utilisateur dans un environnement facilitant l'usage ou l'apprentissage & jeu, formation, réalité virtuelle, ambiance architecturale ou urbaine, installation d'art contemporain \\
		médiateur & rendre effectives les interactions et les transactions & organisation, système de production \\
		évaluatif & aider à la conduite de l'évaluation de performance & système d'évaluation de la performance \\
		quintessenciel & s'adapter en s'informant & contrôleur, logiciel, système mécatronique
	\end{tabular}
	\caption{\cite[Table 4.3, p.77 Typologie des artefacts de classe A]{micaelli_artificialisme:_2003}}
	\label{tab:artefactsdeclasseA}
\end{table}
%}
Nous pouvons donc considérer l'\gls{ACV} comme un artefact de classe A dans les catégories~: final, par défaut ; évaluatif, évaluation environnementale et sociale ; cognitif, repoussant les frontières de rationalité de l'Homme non assisté, non-outillé ; et enfin, quintessenciel, dans sa version conceptuelle finale présentée dans ce mémoire avec une adaptation aux flux d'informations entrant dans les recueils de données avec implémentations sémantiques pour ajuster les choix des divers utilisateurs.

\citeauthor{micaelli_artificialisme:_2003} détaillent dans leur approche mésoscopique de la conception les différentes vues d'un artefact.
\blockcquote[p.132]{micaelli_artificialisme:_2003}{
	axiologique\footnote{\href{http://www.cnrtl.fr/definition/axiologie}{Axiologie}~: Science des valeurs philosophiques, esthétiques ou morales visant à expliquer et à classer les valeurs, CF CNRTL} ; comptable ; conative\footnote{\href{http://gdt.oqlf.gouv.qc.ca/resultat.aspx?terme=conation}{conation},du rapport entre la volonté et l'action, de ce qui perçu comme bon ou mauvais pousse à l'action.} ; conceptuelle ; dimensionnelle ; environnementale ; évaluative ; événementielle ; fabricatoire ; factorielle ; fonctionnelle ; généalogique ; génétique ; morphologique ; nominaliste ; ontologique ; processuelle ; structurelle ; typologique
}
Chacune de ses vues complète la définition de l'artefact (conçu ou à concevoir).
Ils sont autant de plans pour que le système produit réponde effectivement au besoin.
Nous tâcherons de garder en mémoire ces vues lorsque de la reconception de l'ACV.
% reprendre toute la table p132 pour mettre en relation les vues traitées dans la révision de l'ACV.

Enfin, parce que de nombreux éléments théoriques reposent sur le travail de \citeauthor{micaelli_artificialisme:_2003}, nous tenons à prendre des précautions, faute de la distance, dans l'importance accordée par les auteurs au rôle des artefacts avec l'exemple suivant~:
\blockcquote[chapitre 4, p.59 4.2.2 Deuxième exemple: le temps]{micaelli_artificialisme:_2003}{
Le temps est certes une catégorie importante, mais pas évanescente, rétive à notre entendement.
Il est tout simplement impensable et inutilisable sans artefacts de classe A, spécifiquement conçus pour lui donner une subsistance.
Sans ces artefacts de classe A, il n'existe tout simplement pas.}
C'est la dernière phrase qui ici est \emph{inexacte}.
Il y a confusion entre le temps et le concept du temps.
Par exemple, du temps c'est écoulé bien avant la conception des artefacts proposés en exemple~:
\blockcquote[chapitre 4, p.59 4.2.2 Deuxième exemple: le temps]{micaelli_artificialisme:_2003}{
les calendriers, les horloges [\ldots], les fuseaux horaires, les cloches d'églises[\ldots]
}.
Devrions nous donc éviter de dire que la \emph{soutenabilité} n’existe que par les artefacts permettant de la concevoir et de l'évaluer ?
Ce serait faire l'erreur de croire que la soutenabilité a une existence propre (comme le temps) en dehors d’être un \emph{concept} tel que présenté au premier chapitre.

\keybox{La soutenabilité n'existe que par sa conception et son évaluation par des sujets.}

\subsection{Des méthodes de conception}
\label{meth_conception}
Les éléments théoriques précédent intégrés, faisons maintenant un survol des outils d'orientation plus pratique.

%À reprendre
%Modéliser et organiser la conception innovante : le cas de l'innovation radicale dans les systèmes d'énergie aéronautiques


%xxxxxxxxxxxxxxxxxxxxxxxxxxxxxxxxxxxxxxxxxxxxxx
%
%\cite{agogue_introduction_2013}
%Introduction à la conception innovante : éléments théoriques et pratiques de la théorie C-K
%
%xxxxxxxxxxxxxxxxxxxxxxxxxxxxxxxxxxxxxxxxxxxxxx

%Yannou


%valeur, analyse de la valeur

%contexte de recherche du domaine
%\blockcquote{yannou_preconception_2001}{
%Ce domaine est né de la nécessité de conceptualiser la conception qui était quelque peu vue avant les années 1960 comme l’activité d’un « homme de l’art ». Perrin explique fort bien dans [Perrin 2001b] comment dans les années 1960 à 1970, la conception a commencé à être théorisée pour devenir le domaine des design theories et des design methodologies. Ceci a commencé en Angleterre puis aux Etats-Unis et en Allemagne. Quelques professeurs de conception ont théorisé leur savoir dans des ouvrages de référence, ce que nous appellerons les approches générales.
%A part ces approches générales, ce domaine de recherche est riche en thématiques abordées, nous choisirons d’en aborder quelques-unes ici qui nous semblent importantes en conceptual design :
%- l’évaluation multi-critère de solutions,
%- la représentation de l’imprécision en préconception,
%- le lien entre paramètres de conception, performances et besoins.
%En France, très peu de chercheurs nous semblent travailler spécifiquement dans
%ce domaine. Il s’agit d’un domaine majeur de notre thème de recherche et sur lequel nous allons nous recentrer particulièrement dans le futur.}

Dans son \blockcquote[2.3.5]{yannou_preconception_2001}{domaine des méthodologies industrielles de conception},
\citeauthor{yannou_preconception_2001} cite des méthodes les déclarant comme les principales~: \blockcquote[2.3.5]{yannou_preconception_2001}{l’Analyse Fonctionnelle
(AF), l’Analyse de la Valeur (AV), l’Analyse des Modes de Défaillance de leurs Effets et de leur Criticité (AMDEC), le Quality Function Deployment (QFD), la méthode d’innovation TRIZ, la gestion de projet (GdP)}.
Il souligne ensuite que~:
\blockcquote[2.3.5]{yannou_preconception_2001}{en conclusion, les méthodologies citées sont pour la plupart basées sur des textes de norme qui non seulement appauvrissent à escient la complexité du problème, mais aussi ne référencent plus les causes et les hypothèses de leurs préconisations, et se gardent le plus souvent d’indiquer les outils de mise en œuvre. Ces méthodologies progressent plus grâce à leurs associations d’industriels pratiquants et de consultants plutôt que par la recherche universitaire.
%Cela est regrettable car ce sont elles qui influent majoritairement sur les pratiques industrielles.
%Cela est probablement dû à un cloisonnement des personnes qui n’ont pas l’habitude ou le goût de communiquer. Ces associations sont par exemple l’AFAV en France et SAVE aux Etats-Unis en ce qui concerne l’Analyse de la Valeur.
%A cet état de fait, ainsi qu’à celui de la diversité des normes, s’ajoute la multiplication des nouvelles méthodologies industrielles. Citons par exemple, car elles ont un rapport avec notre thème de recherche, les méthodologies suivantes : la Conception à Coût Objectif (CCO), le Management par la Valeur (MV), le Cost As Independent Variable (CAIV), le Earned Value Management (EVA), l’Ingénierie du Besoin ou Requirements Engineering, ...
}
Observons des méthodes pour forger notre propre jugement.
Nous traiterons dans l'ordre l'\gls{AF}, puis les composante de la méthode \textsc{Réseau}.
Nous soulignerons les similitudes des méthodes entre-elles et avec l'\gls{ACV}.
%\url{www.afav.eu/index.php?option=com_content&task=view&id=25&Itemid=39}

%\begin{tcolorbox}[
%					colframe=blue!50!white,
%					width=\textwidth,
%                  %%enhanced,
%                  %%frame hidden,
%%                  interior hidden,
%                  boxsep=2pt,
%                  left=2pt,
%                  right=2pt,
%                  top=2pt,
%                  ]
\subsubsection{Analyse de la valeur et analyse fonctionnelle}

D'après \citeauthor{yannou_analyse_1998},
\blockcquote{yannou_analyse_1998}{le plan de travail d'une action d'\gls{AV} se décompose en 7 phases~:
\begin{enumerate}
 \item Orientation de l'étude
 \item Recherche d'informations
 \item Analyse fonctionnelle et analyse des coûts
 \item Recherche de solutions
 \item Étude et évaluation des solutions
 \item Bilan prévisionnel et choix
 \item Réalisation
\end{enumerate}
%1) Orientation de l'étude
%2) Recherche d'informations
%3) Analyse fonctionnelle et analyse des coûts
%4) Recherche de solutions
%5) Étude et évaluation des solutions
%6) Bilan prévisionnel et choix
%7) Réalisation
}

%xxxxxxxxxxxxxxxxxxxxxxxxxxxxxxxxxxxxxxxxxxxxxxxxxxxxxxxxxxxxxxxxxxx

L'\gls{AF} est donc considérée comme \emph{une} des parties de l'\gls{AV}.
Suivant \citeauthor{philippe_taillard_demarche_2010},
s'appuyant sur les normes (NF X 50-100 ; NF X 50-151 ; FD X 50-101 ; NF EN 1325-1) et la méthode APTE\textregistered, %Nous appliquons la méthode APTE~\cite{ouvrage de Cosmin + AFAV}.
l'\gls{AF} est décrite comme se déroulant de la façon suivante~\cite{philippe_taillard_demarche_2010}~:

\begin{enumerate}
\item \gls{AB}~:
Cette étape consiste en la verbalisation du besoin et ses relations au produit à concevoir ainsi qu'à la matière d’œuvre modifiée. La représentation graphique consacrée est le 'schéma du besoin'.
\item \gls{AFB}~:
Cette étape consiste en la traduction sous forme de services générés nécessaires à la satisfaction du besoin par l'usage du produit.
L'étape est constituée des éléments suivants~:
	\begin{enumerate}
	\item Identification des phases de vie du produit
	\item Pour chaque phase de vie~: 
		\begin{enumerate}
		\item Identification et caractérisation des \glspl{EME} 
		\item Identification des \gls{FS} 
		\item Caractérisation des \gls{FS}
		\end{enumerate}
	\item Rédaction du cahier des charges fonctionnel~: Ce type de document comprend l'identification, la caractérisation et la validation des \glspl{FS} et \glspl{EME}.
	\end{enumerate}
\item \gls{AFT}~: Ce travail consiste en schéma bloc de lecture bidirectionnel horizontal pour répondre au question Comment $\rightarrow$ ? ; Pourquoi $\leftarrow$ ? et naviguer des besoins fonctionnels exprimés en fonctions au solution techniques envisagées pour y répondre\footnote{Les FAST sont parfois également utilisés pour la `lecture' de système existant. i.e. l'observation des composants et des fonctions auxquels ils répondent.}.
\end{enumerate}

\subsubsection{RÉSEAU}
\label{subsubsec:RÉSEAU}

Nous notons une forte similarité à la description du recueil de méthodes RÉSEAU décrit par \citeauthor{tassinari_pratique_1997}~:
\blockcquote[Méthode RÉSEAU (11)]{tassinari_pratique_1997}{
\begin{enumerate}
\item \textbf{R}echerche intuitive
\item \textbf{É}tude du cycle de vie et de l'environnement
\item \textbf{S}equential Analysis of Functional Elements (SAFE)
\item \textbf{E}xamen des mouvements et efforts
\item \textbf{A}nalyse d'un produit de référence
\item \textbf{U}tilisation des normes et règlements
\end{enumerate}
}

Au sein de cette sélection de méthodes, les descriptions de chaque 'module' sont données.
%Recherche intuitive
%Étude du cycle de vie et de l'environnement
%Sequential Analysis of Functional Elements (SAFE)
%Examen des mouvements et efforts
%Analyse d'un produit de référence
%Utilisation des normes et règlements
%En particulier des méthodes de ce réseau nous semble quasiment identiques.
Nous notons la similarité avec la description de la Recherche intuitive.
Celle-ci comporte les étapes suivantes~:
\begin{itemize}
\item  Rappel des objectifs (fiche-programme ; marketing ; entrepreneurial),
\item  Documentation, examen préalable,
\item  Recherche des fonctions (énoncées tout venant sans classification),
\item  Critique, soit un débat sur les fonctions,
\item  Formulation (reformulation des fonctions retenues après débats),
\item  Définition des caractéristiques (critères, niveaux, flexibilité) pas de hiérarchisation à ce stade,
\item  \gls{CdCF} (rédaction du cadre, de la liste de fonctions caractérisées à compléter par les autres étapes de réseau).
\end{itemize}
De même l'"Étude du cycle de vie et de l'environnement" est équivalent à la recherche des \gls{EME}

La méthode \gls{SAFE} consiste à se mettre à la place de l'utilisateur et à décomposer les séquences d'actions.
(i) Tâche : ensemble de séquences,
(ii) Séquence : ensemble d'opérations,
(iii) Opération : ensemble de phases,
(iV) Phase : unité d'action.
Sur la base de la description de la séquence d'actions, il faut ensuite exprimer des fonctions.
\exbox{
Nous reprenons l'exemple donné dans l'ouvrage de \citeauthor{tassinari_pratique_1997}, celui de l'aspirateur~\cite{tassinari_pratique_1997}~:\\
Chercher l'aspirateur, saisir l'aspirateur, déplacer l'aspirateur.\\
Être identifiable, être préhensible, être maniable et être peu encombrant.}
Suite à l'expression des fonctions celles-ci doivent être caractérisées pour compléter le \gls{CdCF}.
%SAFE = faire référence au cycle de vie défini précédemment 
%faire référence aux EME défini précédemment 
%sélection les phases de vie méritant décomposition en séquence
%imaginer / observer la gamme opératoire
%établir le graphe des séquences
%rechercher les fonctions à partir des séquences
%formuler les fonctions
%caractériser les fonctions
%compléter le CdCF

Concernant l'\emph{Examen des mouvements et des efforts},
nous notons ici une divergence avec les auteurs.
La démarche comprend l'observation des fonctions techniques (soutien nécessaire des fonctions de services), des mobilités et contraintes de dimensionnement pour des sollicitations habituelles, occasionnelles ou exceptionnelles.
\blockcquote{tassinari_pratique_1997}{
Pour ce qui se rapporte aux produits du type processus administratif ou informatique, il est évident qu'ils ne sont pas concernés par les efforts et que seuls les mouvements sont à prendre en compte.
}
Les auteurs négligent ici les efforts psychologiques et cognitifs~:
Persévérance dans une tâche longue ; Renonciation à l'erreur, coûts de son admission et sa déclaration préalable à sa correction ; Limites des connaissances et des capacités calculs
% ex : retenir une liste de 632 Noms et Prénoms à la première lecture sans support autre que la mémoire
\ldots

Cette description de la méthode nous incite d'ailleurs à expliciter ces efforts dans notre application de la conception (Chap.~\ref{chap:ACV, la (re)conception d'un outil}) en correspondance à l'\textbf{Analyse d'un produit de référence} (l'ACV actuelle pour nous).
%Examen des mouvements et des efforts = faire référence au cycle de vie défini précédemment 
%faire référence aux EME défini précédemment 
%Rechercher les composants concernés (générateurs, transmetteurs, transformateurs, interrupteurs, récepteurs)
%caractériser les mouvements (genre, amplitude, cycle, enchaînement, temporisation)
%Identifier la nature des efforts : action interaction réaction
Dans l'analyse d'un produit de référence,
\blockcquote{tassinari_pratique_1997}{
les questions posées à l'utilisateur peuvent concerner ses insatisfactions.
[\ldots]

l'ergonomie, la sécurité, la fiabilité, la disponibilité, la maintenabilité...
[\ldots]

Ce double examen fonctions - solutions permet de situer sa propre étude.
}

Nous le voyons donc, ces méthodes servent à identifier et stabiliser les socles quantifiables et qualifiables de l'artefact à produire, à le caractériser.
Cette base sera ensuite employée comme référence (cible, objectif) dans des démarches de rationalisation. 
%Utilisation des normes et règlements
%= fonctions contrainte réglementaire
\subsubsection{Des méthodes critiquées}

Dans \citetitle{darses_francoise_assister_2001}, \citeauthor{darses_francoise_assister_2001} discutent les méthodes de conception et leur adéquation aux processus cognitifs de conception.

\blockcquote{darses_francoise_assister_2001}{On a constaté (Darses, 1997 ; Nicolas, 2000) que l’analyse fonctionnelle entre en contradiction avec les processus cognitifs effectivement mis en œuvre par les concepteurs.}
Les auteurs semblent toutefois plus faire la critique des guides et méthodes 'standardisées' critiquées également par \citeauthor{yannou_preconception_2001}.
Le fond de leur critique reposant sur ce qu'elle (l'\gls{AF}) proposerait
% En effet,
\blockcquote{darses_francoise_assister_2001}{
l'idée d'accomplir la phase d’analyse du problème avant de passer à la phase de recherche de la solution [et donc de ne tenir compte] de l'activité cognitive de résolution des problèmes de conception [\ldots] qui s'opère par itération des phases « définition de problèmes »/ « élaboration de solutions ».
}
L' \textit{itérativité} reste toutefois décrite dans la littérature, comme présentée dans ce chapitre.

Il reste qu'il existe de nombreuses méthodes, complémentaires et donc individuellement incomplètes.
D'où notre présentation de \textsc{Réseau}, comme recueil de méthode.
Mais puisque nous avons vu la similarité de l'\gls{ACV} et de la conception, gardons à l'esprit la critique que nous ferons de la première (ACV) lorsque nous repenserons à la seconde (conception).
Si le choix parmi des alternatives nécessite la construction d'un modèle de valeurs, la construction d'un modèle de valeur nécessite peut-être d'être confronté à des choix.

\subsection{Conclusion sur la conception}
Nous avons donc vu la proximité et l'imbrication des processus évaluatifs, décisionnel et de conception.
Mettant en œuvre le processus et les outils méthodologiques de la conception, nous tâcherons d'apporter une attention particulière sur les vues et les niveaux d'architecture qui nécessitent selon nous la reconception de l'\gls{ACV}.

\keybox{
Puisque l'ACV en appelle au caractère fonctionnel des systèmes qu'elle étudie, il apparaît de façon évidente que \textbf{toute application de l'ACV requière celle d'outils de la conception.}
}
Nous avons également constaté que le panel d'outils sert une complémentarité.
Les critiques et difficultés relatives aux unités fonctionnelles et périmètres sont donc liées à la critique de l'artificilisme.
Comme il en a été fait la critique pour les méthodes de conception, il nous faudra nous garder d'enfouir `la complexité des problèmes[\ldots] les causes et les hypothèses des préconisations' ainsi que de conserver un corps universitaire vif plutôt que de faire l'abandon de l'outil à des `associations d'industriels pratiquants et de consultants'

Pour chacun de ces domaines, \gls{ACV}, \gls{ADMC}, artificialisme, la question des valeurs se retrouve dans cette recherche de~: \textit{"Qu'est-ce qui nous (in)satisfait et dans quelle mesure ?"}.
Au regard de leur similitude, prenons pour communes les critiques que nous ferons de l'\gls{ACV} ainsi que des méthodes de conception et d'évaluation, qui sont en fait des processus imbriqués.
L'éco-conception n'est dans ce cadre qu'une orientation particulière de l'évaluation pour la conception avec la prise en compte de paramètre connotés 'éco'.

Notre parcours théorique des archétypes Rationaliste et Traditionaliste nous aura éclairé~:
\begin{itemize}
\item d'une part sur les orientations actuelles de l'ACV (attentisme et constructivisme radical), dont nous retrouvons les orientations dans les pratiques et les normes,
\item d'autre part sur la tendance traditionaliste du rejet de la complexification de l'\gls{ACV}, voir du rejet de l'\gls{ACV} elle-même\footnote{
\blockcquote{bare_life_1999}{
Certains ont préconisé l'abandon complet de l'ACV, car elle n'est pas réalisable dans sa forme la plus sophistiquée[\ldots]
%Some have advocated abandoning LCA altogether, since it is not achievable in its most sophisticated form.
}
}.
\end{itemize}

Enfin le traitement théorique de ce chapitre souligne des éléments capitaux~:
\begin{itemize}
\item Au dela de l'absence d’existence \textit{objective} d'état de soutenabilité, il n'y a de soutenabilité qu'au travers de l'évaluation.
\keybox{
La soutenabilité est le produit de l'activité d'évaluation.
}
La soutenabilité est donc la mise en œuvre d'un processus itératif.
\item La difficulté d'abandon de la doctrine 'one best way' entrave la science de la conception autant que l'\gls{ADMC} et l'\gls{ACV}.
\keybox{
Les méthodes de conception comme celles de ces disciplines sœurs ne font pas apparaître le traitement du jugement de valeurs.
}
\end{itemize}
Nous pouvons affirmer qu'il n'existe pas de `meilleure alternative' et cela sans requérir d'approche de la \textit{théorie de la contingence}, (i.e. la meilleure alternative dans le contexte X n'est pas celle du contexte Y).
L'absence de 'meilleur' et la seule existence du 'préféré' réside dans l'incommensurabilité.
\keybox{
Les problématiques liées à ce fait en conception sont donc également traitées par notre travail.
}
