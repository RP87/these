\section*{Lecture du document}

Afin de guider et fluidifier la lecture, les principes suivants sont établis.
\exbox{Le cadre des exemples illustrant les propos sont dans ce format.}
\figbox{La lecture spécifiquement des figures, de certaines équations et tables est faîte dans cette type d'encadrement.}
\keybox{Les éléments clefs sont encadrés de cette façon.}

Le document comporte des outils pour une lecture numérique.
Lors qu'il l'a été jugé adéquat, des liens vers le web sont présents (\href{https://duckduckgo.com}{avec ce format}).
Les références (bleues entre crochets) vous conduisent à leurs localisations dans la bibliographie.
Là un hyperlien rouge interne pointera par son numéro la page d'insertion (et donc votre origine pour y retourner aisément).
L'ensemble du sommaire comme les termes du glossaire sont liés aux termes qu'ils mentionnent.
Comme pour la bibliographie, les termes du glossaire dans le corps de texte sont liés au glossaire et les numéros de pages vous guident pour le retour.
Les références faites aux chapitres, sections et notes sont également actives.
