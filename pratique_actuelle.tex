\paragraph{FS1} 
Présenter la préférence du décideur envers une série d'alternatives, suivant la globalité de son système de valeurs.

Actuellement, l'\textit{évaluation} réside dans la caractérisation, issue d'observation ET de jugements opérés de façon indépendants des interprétations finales.

À l'issue de la reconception, la sortie finale est un réseau hiérarchisé avec préférences flous d'alternatives en accord au système de valeur du décideur.

\paragraph{FS2} 
Respecter les limites cognitives humaines durant son processus et permettre de les dépasser.

Dans son état actuel, l'ACV repose sur des heuristiques d'application humaine à la sélection des indicateurs, à la détermination des unités fonctionnelles, à l'issue des résultats 'caractérisés'.
Les praticiens comme les 'clients' sont dans l'incapacité de réaliser le contrôle des jugements.
Ce qui n'est d'ailleurs même pas un soucie identifié par la communauté.

Après cette reconception, la capacité de contrôle de la consistance du jugement appliqué devient centrale.
Il y a réduction du nombre d'informations à post-traiter humainement sous le seuil de surcharge cognitive.

 
\paragraph{FS3} 
Être adaptée aux distributions géographiques, linguistiques et culturels des parties prenantes.

Localisation est pour le moment non-systématique.
La littérature est majoritairement anglophone.
Les jugements sont principalement occidentaux.
Quelques développements localisés existent sur les méthodes d'impacts (LUCAS, Canada ; LIME, Japon), non nécessairement traduits\footnote{\blockcquote{jrc_ilcd_2011}{ Information in non-Japanese language only partially available.}}.
Il n'y a pas d'information qui assiste l'application d'outils automatiques de traduction.
Pas de traitement des multiples orientations culturelles.
L'emploi de données spécifiques et à la charge justificatrice du praticien dans l'exception.

Après ce cycle de reconception la localisation des observations est systématique.
L'ouverture à la lecture et l'écriture permettent la traduction manuelle.
L'implémentation sémantique permet la traduction par script.
 
\paragraph{FS4} 
Assurer son développement en données.

C'est un problème récurant depuis la genèse jusqu'à l'état actuel.
Rien dans la méthodologie n'est posé pour sa résolution.
Des approches sémantiques sont en projets~\cite{weidema_bonsai_2014,vardeman_ontology_2015}, quelques données dans des bases sont libres en lectures.
 
Après reconception, les mécanismes de l’open-source associé à la production par scriptes font partie intégrante de la production en ACV.
La production d'évaluation est la résultante directe d'une interrogation de la base de connaissance depuis un système de valeur incluant des contraintes de qualité sur le matériel interrogé.


\paragraph{FS5} 
Assurer sa validité et crédibilité. 

La crédibilité de l'ACV est mise en défaut par une série de critiques non résolues et de multiple jugements de valeurs introduit dans la méthodologie.
La critique de la qualité des données et d'un traitement de l'incertitude n'est pas systématique aux études d'ACV.
Nous relevons une présence importante de donnée sans distribution\footnote{\blockcquote{mutel_why_2013}{ecoinvent 2.2 database Distribution Undefined	Technosphere~:~1629 ; Biosphere~:~37036}}.
Il y a de grande difficulté de traçabilité de la donnée.
 
De par notre reconception, le traitement des jugements de valeurs est identifié, isolé, explicite.
Les dépôts de données sont nominatifs, datés, localisés.
L'emploi des agrégations se fait par catégorie, sans "création" de données nouvelles (synthèse via la classification sémantique, via l'ontologie).
Le mécanisme de revue par les pairs par tag sémantique contribue à l'adéquation entre la qualité de l'information et les critère du décideur.
Il y a continuité de la responsabilité des auteurs.


\paragraph{FS6} 
Respecter les contraintes réglementaires s'imposant à elle.

Le copyright\textcopyright domine en présence. S'en suit le respect de la propriété de données majoritairement en 'biens de clubs' conduisant soit à l'inaccessibilité pour une vaste majorité des parties prenantes, soit par une violation des droits d'auteurs pour la consultation illégale.

Le reconception de l'ACV conduit à ce que les données soient systématiquement en open access.
Pas de copyright\textcopyright, donc pas de violation des droits de propriété lucrative.
Le contrôle sur les droits d'auteurs est possible par le contrôle de l'auteur du dépôt des descriptions des systèmes industriels ou environnementaux, ou des systèmes de valeurs anonymisés et collectifs recueilli par enquêtes (toutes les contributions sont nominatives). 
 
\paragraph{FS7} 
Protéger les déclarants sur l’énoncé de leurs jugements. 

L'absence des déclarations rend nul ce point dans l'état de l'art actuel.

Après reconception, l'anonymisation pour les requêtes via les systèmes de valeurs individuels est requise.
Il y a nécessité de protection pour le recueil des matrices collectives pour ne pas ciblé d'individu dans l'agrégat collectif.


%Confrontons donc de façon synthétique les caractéristiques décrites à l'issue de l'\gls{AF} et l'état actuel dans la table~\ref{tab:issue-actuel}.
%\begin{table}
%\begin{longtable}{|c|p{3cm}|p{4.5cm}|p{4.5cm}|}
%\hline
%Libellé & fonction & Issue \gls{AF} & État actuel\\
%\hline
%FS1 & Présenter la préférence du décideur envers une série d'alternatives, suivant la globalité de son système de valeurs.& Réseau hiérarchisé avec préférences flous d'alternatives en accord au système de valeur du décideur & Évaluation~: caractérisation issue d'observation ET de jugement opéré de façon indépendante des interprétations finales\\
%FS2 & Respecter les limites cognitives humaines durant son processus et permettre de les dépasser.& Capacité de contrôle de la consistance du jugement appliqué. Réduction du nombre d'informations à post-traiter humainement sous le seuil de surcharge cognitive. & Heuristiques d'application humaine à la sélection des indicateurs, à l'issue des résultats 'caractérisés'. Incapacité au contrôle des jugements\\
%FS3 & Être adaptée aux distributions géographiques, linguistiques et culturels des parties prenantes.& Localisation des observations systématique. Ouverture à la lecture et l'écriture permettant et donc la traduction manuelle. Implémentation sémantique permettant la traduction par script. J& Localisation non-systématique.  Littérature majoritairement anglophone, jugements occidentaux. Quelques développements localisés sur les méthodes d'impacts (LUCAS, Canada ; LIME, Japon), non nécessairement traduits\footnote{\blockcquote{jrc_ilcd_2011}{ Information in non-Japanese language only partially available.}}. Pas d'information assistant l'application d'outils automatiques de traduction. Pas de traitemendt des multiples orientations culturelles.\\
%FS4 & Assurer son développement en données.& Mécanismes de l’open-source associé à la production par scriptes & Problème récurant depuis la genèse jusqu'à l'état actuel. Approche sémantique en projets~\cite{weidema_bonsai_2014,vardeman_ontology_2015}. Quelques données dans des bases libres en lectures~\ref{subsect-bases_ouvertes, ex agribalise, carbone, impact}\\
%FS5 & Assurer sa validité et crédibilité. & Traitement des jugements de valeurs identifié, isolé, explicite. Dépôts de données nominatifs, datés, localisés, emploie des agrégations par catégorie sans "création" de données nouvelles (synthèse via la classification sémantique, l'ontologie). Mécanisme de revue par les pairs par tag sémantique. Continuité de la responsabilité des auteurs. & Critique non résolue de multiple jugements de valeurs introduit dans la méthodologie. Critique de la qualité des données et d'un traitement de l'incertitude non-systématique. Présence importante de donnée sans distribution\footnote{\blockcquote{mutel_why_2013}{ecoinvent 2.2 database Distribution Undefined	Technosphere~:~1629 ; Biosphere~:~37036}}. Difficulté de traçabilité de la donnée.\\
%FS6 & Respecter les contraintes réglementaires s'imposant à elle. & Données systématiquement en open access. Pas de \textcopyright, donc pas de violation des droit de propriété. Contrôle possible sur les droits d'auteurs par le contrôle de l'auteur du dépôt (nominatif). & \textcopyright, respect de la propriété de données majoritairement 'biens de clubs' conduisant soit à l'inaccessibilité pour une vaste majorité des parties prenantes, soit par une violation des droits d'auteurs pour la consultation illégale.\\
%FS7 & Protéger les déclarants sur l’énoncé de leurs jugements. & Anonymisation pour les requêtes via les systèmes de valeurs individuels. Nécessité de protection pour le recueil des matrices collectives. & Absence des déclaration prépondérante. Non concerné.\\
%\hline
%\caption{Confrontation état initial, reconception}
%\label{tab:final-actuel}
%\end{longtable}
%%\end{table}