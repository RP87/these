Comme constaté en littérature, \emph{la méthodologie de l'\gls{ACV} s'avère défectueuse}.
Nous avons donc entrepris tout d'abord pour répondre à l'\textbf{interrogation générale} précédente, de débuter par l'\textbf{interrogation particulière} suivante~:
\textbf{Quelle(s) est (sont) la (les) cause(s) de la défectuosité de la méthodologie de l'ACV~?}
Nous commençons donc par la question de la soutenabilité, ses divers concepts et les outils d'évaluation produits jusqu'ici (\ref{chap:Évaluation de la Soutenabilité, un État de l'art}).
Nous traitons ensuite des outils pratiques et théoriques (\ref{chap:Méthodes et outils}), spécifiques à l'\gls{ACV} d'abord (\ref{sec:Outils de modélisation en ACV}), puis généraux à la conception (\ref{sec:Méthodologies et théorie de la conception})et à la décision multi-critère (\ref{sec:ADMC}).

L'ACV est un artefact, produit pour répondre au besoin de rationalisation étendue de décisions humaines avec une finalité orientée sur la soutenabilité.
Nous avons donc mobilisé dans un chapitre de reconception de l'ACV (\ref{chap:ACV, la (re)conception d'un outil}) outils et théories de la conception~\ref{sec:Méthodologies et théorie de la conception} et de la décision~\ref{sec:ADMC}, pour traiter de la soutenabilité~\ref{sec:La Soutenabilité}.

La lacune majeure de l'\gls{ACV} réside dans l'absence de la prise en compte explicite et consistante du \emph{jugement de valeurs du décideur} dans cet outil d'aide à la \emph{décision}.
Ceci induit donc le questionnement réciproque~:\\
\textbf{Comment intégrer le jugement de valeurs du décideur dans une approche rationalisée de la décision pour la soutenabilité~?}

La reconception méthodologique de l'ACV (chap.~\ref{chap:ACV, la (re)conception d'un outil}), nous permettra d'introduire cela avec \emph{l'un des thèmes les plus controversés de la discipline}, \textbf{la résolution du traitement de la multi-fonctionnalité} (chap.~\ref{chap:Multifonctionnalité}).
La généralisation apportée par la \emph{théorie unifiée de l'allocation} permet l'application de nos travaux au versant économique de cette problématique.

L’extension de ces éléments au traitement de la multi-dimensionnalité aboutie la démarche et ouvre une voie hors ACV (chap.~\ref{chap:Jugements et Multi-dimensionnalité}).
Notre contribution souligne la tentative de domination de la valeur économique, du profit et de la pensée attenante.
Elle entérine donc au passage qu'il n'y a d'économie que l'économie politique, au sens de sa critique marxienne.

Les \href{http://www.cnrtl.fr/definition/doctrine}{doctrines} philosophiques reconnaissant un ordre axiologique pré-établi sont rejetées ici.
Produite par un \href{http://www.cnrtl.fr/definition/agnostique}{agnostique}
\footnote{Lien vers l'\href{http://www.cnrtl.fr/definition/agnosticisme}{agnosticisme}}
cette contribution n'exclue ni n’empêche tout partisan d'un dogme de formuler \emph{sa} structure \href{http://www.cnrtl.fr/definition/axiologie}{axiologique} afin de l'employer avec notre méthodologie. % dans \emph{ses} propres évaluations.
L'emploi de notre méthode n'est donc pas conditionné par un acte de foi préalable, mais s’accommode de ceux que le décideur se résout à prendre.

La seconde défectuosité que nous traitons réside, non pas dans l'\gls{ACV} mais dans l'organisation du travail au sein du milieu académique lui même.
Obstacle, si moins controversé, tout aussi majeur, la qualité et la disponibilité des données est un point de \textit{récurrence systématique} de notre domaine.
\textbf{Puisqu'il n'est possible de prendre de décision rationnelle que sur l'espace connu de la réalité, étendre la rationalité repose sur l'extension de cet espace connu.}
Tout obstacle à l'extension de l'espace connu est un obstacle à la rationalité.
Or il s'avère que l'appareil de dissémination de la connaissance scientifique en est majoritairement resté à l'âge de l'imprimerie.
Nous avons donc cherché~:
(i) Pourquoi le monde académique n'a pas fait évoluer ses pratiques et outils de dissémination ? (ii) Comment le lui permettre ?
Ceci, d'une portée plus large que l'ACV même si nécessaire à cette dernière, est traité au chapitre~\ref{chap:Recherche Libre}.