\section{Cadre de la recherche}
%\keybox{
S'il s'agit de Recherche, alors les acteurs ne peuvent anticiper ce qu'ils trouveront.
S'ils le peuvent, ils font alors du développement et non de la recherche.
%}
Cela ne nous aura pas été aisé, mais nous fîmes de la recherche.
%Et avec elles nous eûmes son lot de surprises. 
N'eussions nous pas fait de recherche, que nous n'aurions pas eu de surprises dont certaines qui vous seront présentées dans ces pages.

Je considère de façon poperrienne que la science avance par falsification.
C'est toutefois un processus parfois douloureux.
Étant moi même issu d'une formation en ingénierie et conception industrielle, le 'one best way' à mis un certain temps (et une certaine douleur) a être révoqué.
J'ai pris durant ces trois années certaines de mes idées pré-conçu dans la figure.
J'espère que les lecteurs en tirerons les bénéfices avec une moindre douleur.

Parce que je ne crois pas à la science objective, la science étant l'œuvre de sujets, j'en suis venu à la conclusion qu'en toute honnêteté, il me fallait décrire (i) mon cadre (ii) mes orientations (iii) faire l'emploie de la première personne lorsque cela est adéquat.
Voyez donc ceci comme une sorte de déclaration d'intérêts étendue.
\subsection{Un emploi et des tâches}

Le projet MOld innovaTION (MOTION) financé via un Projet Investissement d'Avenir à hauteur de 1.7m€ et sous le contrôle de l'\gls{ADEME} sert le développement de la société Arc International pour \blockcquote{ademe_motion._2013}{améliorer sa compétitivité et [mettre] en place [\ldots] une démarche d’éco-conception [\ldots] des outillages utilisés en verrerie.}

À mon recrutement par l'École Centrale de Lille (ECLi), l'intitulé initial de la thèse et de son objet étaient définis comme suit~:\\
``Étude du recyclage des copeaux et/ou des moules métalliques utilisés en verrerie - procédés, propriétés des pièces et empreinte environnementale.''
\footnote{
CF \href{http://umet.univ-lille1.fr/details.php?show=370&lang=fr}{site de l'UMET}. Ceci sera certainement un lien mort après mon départ de l'université.
}
%\subparagraph{Conduite d'ACV, découpage en lots}
Le lot 1 "éco-conception", attribué à mon employeur (ECLi) et donc à moi même, comportait un certain nombre de tâches.
Leurs libellés étaient les suivants~:
\begin{itemize}[noitemsep]
\item ``31 Bibliographie des études réalisées, des méthodes employées et des données disponible ;
\item  \textbf{32} Création d'un outil de modélisation du cycle de vie actuel des moules ;
\item  \textbf{33} Modélisation du cycle de vie actuel des moules et analyse environnementale ;
\item  \textbf{34} ACV sur les critères principaux des étapes du cycle les plus impactantes ;
\item  38 Validation / Réorientation / Priorisation des axes de travail ;
\item  \textbf{39-41} ACV des solutions techniques et aide aux choix techniques ;
%LOT 1 40 ACV des solutions techniques et aide aux choix techniques ;
%LOT 1 41 ACV des solutions techniques et aide aux choix techniques ;
\item 47 Proposition de nouveaux axes de travail ;
\item 48 Validation et chiffrage de la diminution des impacts environnementaux.''
\end{itemize}

La phraséologie est caractéristique d'une certaine approche de ce type de recherche.
Mais il est beaucoup plus facile de le voir aujourd'hui, à l'écriture de ce mémoire de thèse qu'à son initiation.
``ACV sur les \textbf{critères principaux des étapes du cycle les plus impactantes}."
``\textbf{Validation et chiffrage} de la \textbf{diminution des impacts environnementaux}".
Les suppositions sont donc~:
(i) Un ordre \textit{naturel} des valeurs avec pour conséquence la capacité de statuer sur la hausse ou la baisse au sein de cet ordre, (ii) l'existence d'une alternative intrinsèquement meilleure qu'il convient d'identifier et d'appliquer pour ensuite valider et quantifier les conséquences\footnote{cf étude des courants antérieurs à l'artificialisme au~\ref{subsection:Éléments théoriques}}.
Ces deux positions sont balayées au premier et second chapitre.
Il ne reste donc plus grand chose des fondements initiaux de ce lot.

Le choix de la méthode d'évaluation environnementale, l'ACV, tout à fait pertinent, tenait donc des orientations pré-établies du projet.
Contrairement aux expressions des tâches du lots ci-dessus, ce choix ne nous pose pas de problème outre mesure.
Nous ne faisons pas ici la négation de la conclusion de l'\emph{in}-opérationalité de l'ACV (cf \ref{subsec:Historique de la méthode, son développement, son contexte}), traitée au chapitre ~\ref{chap:ACV, la (re)conception d'un outil} pour le diagnostic actif et \ref{chap:Multifonctionnalité}, \ref{chap:Jugements et Multi-dimensionnalité} et \ref{chap:Recherche Libre} pour la résolution.
\emph{Entendons nous bien sur la justification de la sélection de l'outil.
L'\gls{ACV} n'aurait pas été proposée que nous l'aurions nous même sélectionnée.}
Si nous pouvons affirmer que l'outil relativement à sa fonction n'est pas abouti, il reste pour sa fonction le plus avancé.

Parce que les procédés discutés dans le cadre des activités d'AIF sont confidentiels, que \emph{toute confidentialité s'accompagne d'un délai} et que consécutivement l'information finie par être ouverte,
nous n'aborderons que de façon générale le milieu de la verrerie et de la métallurgie comme le cadre d'application initial de la méthode.
%Nous traiterons également de la conduite d'un projet de recherche en ACV et de notre perception de la puissance publique dans ce cadre.

\subsection{Verrerie et métallurgie, contexte}
La démarche d'ACV démarre comme vu dans la section~\ref{sec:La pensée en cycle de vie} par le cadrage du but de l'étude et de son périmètre.
Selon les normes actuelles, selon l'envergure du cadre décisionnel d'étude, des variantes méthodologiques sont à appliquer (étude attributionnelle d'aide à la décision à l'échelle micro-économique, étude conséquentielle d'aide à la décision à l'échelle macro-économique.)
Nous allons donc observer notre position relativement à l'importance des flux dans les chaînes de valeur des secteurs de la métallurgie et de la verrerie.

Notre étude d'\gls{ACV} s'inscrit dans le cadre de l'éco-conception de moules de verrerie, \href{http://www.ademe.fr/motion-econconception-recyclage-doutillage-verrerie}{comme publiquement stipulé}.
Notre contexte de travail vise donc les pièces métalliques pour la fabrication d'articles en verre.
Au travers du \citetitle{maria_best_2013}, relatif à la production d'articles verriers~\cite{maria_best_2013}, nous retiendrons les éléments suivants.

Les principaux procédés d'obtention de verres dits domestiques, à usage des arts de la table, sont
\footnote{
Sur cette liste, des hyperliens de vidéos illustratives sont disponibles pour les lecteurs du PDF.
Nous jugeons ces vidéos utiles dans ce qu'\textbf{elles illustrent l'inter-relation, l'imbrication complète, de l'objet d'étude à son contexte fonctionnel.}
Nous considérons qu'il en va de même pour tout produit ou service.
} :
\begin{itemize}
\item la \href{https://www.youtube.com/watch?v=HQw2rP-yffU}{centrifugation},
\item le \href{https://www.youtube.com/watch?v=f4q5RphNFwo}{pressage},
\item le \href{https://www.youtube.com/watch?v=o_HqEeMmrfI}{pressage-soufflage}
\footnote{
D'autres vidéos~: sous la désignation \href{https://www.youtube.com/watch?v=SypruPV9CNw}{H28}, et la version \href{https://www.youtube.com/watch?v=Dqo06N7eofA}{Taïwanaise JCL}
}
sous ces versions~:
\begin{itemize}
\item \href{https://www.youtube.com/watch?v=ixAIrgkHnIo}{pressé-soufflé-fixe}
\item \href{https://www.youtube.com/watch?v=FsVtEcqe4co}{pressé-soufflé-tourné}\footnote{ Vidéo italienne avec complément d'assemblage de pied.},
\end{itemize}
\item le soufflage-soufflage.
\end{itemize}
%Correspondant aux figures respectives~:
Les procédés de soufflage sont également les modes de fabrication dominants des contenants (du flaconnage).
\textbf{Une étude couvrant les procédés de la famille des verres domestiques} (3,86\% de la production de l'industrie du verre en Europe en 2005, 1,46 millions de tonnes), \textbf{couvre donc les procédés de la famille des verres de contenants} (53\% 2005 EU-25, 20 millions de tonnes).
Nous pouvons donc dire que si l'étude de la verrerie pour les arts de la table ne couvre pas la majorité de la production verrière européenne, elle couvre la diversité des procédés qui font la majorité de la production européenne.
%AIF n'emploie cependant plus le soufflage-soufflage et nous ne connaissons pas pour le moment la part de ce procédé au sein du flaconnage.

Le ``best available techniques'' (BAT) nous informe également sur les consommations et rejets moyens.
Pour la famille des contenants et verres domestiques, les consommations moyennes de la catégorie ``Moules et autres'' sont respectivement de \textbf{5 et 2~kg par tonne de verre fondu}.
Cela représente donc pour les procédés étudiés, sur la base de la production européenne, un ordre de grandeur de 100~kt.
Considérons que la majorité des matériaux de la classe ``Moules et autres'' soient des aciers (ce qui n'est pas le cas), il faut mettre ce tonnage en perspective avec l'ensemble des applications.
L'ouvrage \citetitle{remus_best_2013}, relatif à la production d'acier~\cite{remus_best_2013} fait état pour le groupe EU-27 d'une production annuelle de 130,000 kt.

L'exploration du dataset d'ecoinvent sur la production verrière confirme cette position.
\textbf{L'acier inoxydable aux intrants de ce dataset génère une contribution très faible sur l'ensemble du processus.}
Ceci d'après le calcul via Simapro sur les données ecoinvent.
\figbox{
Dataset 'Packaging glass, white {GLO}| market for | Alloc Def, U' ; par la méthode~: IMPACT 2002+ V2.05 ; employant les facteurs de normalisation~: IMPACT 2002+ ; résultat exprimé par l'indicateur~: Score unique.
La contribution de l'acier inoxydable s'élève à 0,0045\% du score final\footnote{
Ce résultat, comme tous ceux que nous avons pu voir ou produire jusqu'ici en termes d'évaluation environnementale, au regard de la critique méthodologique faites dans les chapitres suivants,
(i) ne veulent pas dire \textit{grand} chose,
mais (ii) ne sont pas nécessairement vide de sens.
Sans négliger toutefois que ce dernier pourrait parfaitement nous induire en erreur.}.
Toujours sur le même calcul, nous obtenons une contribution de~:
\begin{itemize}
\item 0,01\% pour la santé humaine,
\item 0,00148\% pour le réchauffement global,
\item 0,014\% pour la qualité des écosystèmes.
\end{itemize}
}

Par ailleurs, l'article de \citeauthor{gros_recyclage_2007}, \citetitle{gros_recyclage_2007} montre une part dominante des produits en fin de vie (52~\%) et des chutes sidérurgiques (29~\%).
Dans la catégorie chutes de transformation (les 19~\% restant), la discussion est orientée sur les grands domaines, 
\blockcquote{gros_recyclage_2007}{de 40 à 45~\% de chutes en emboutissage de carrosseries automobiles à quelques pour-cents dans la construction métallique}.
Il n'est donc que peu probable qu'une approche marginale modifie en quoi que ce soit le paysage actuel et ses caractéristiques (impacts comme chaîne de valeur).
Pour l'affecter, il faudra `taper fort' dans les habitudes et les procédés.

\keybox{
C'est à ce titre que nous estimons le cadre d'étude comme sans incidence majeure sur la chaîne de valeur de la métallurgie.
%Il nous faudra toutefois prendre le recul des pièces usinées de façon générales.
Nous étions donc, sur la base de cette étude préliminaire, tant sur le plan économique qu'environnemental dans un cas d'échelle micro d'aide à la décision.
}

C'est sur cette base que nous avions approcher l'\gls{ACV} et sa méthodologie.
De telles contributions relatives, quasi nulles, ne doivent pas induire en erreur le praticien, la praticienne, qui les rencontrerait sur son cas d'étude.
Il convient de rester attentif sur le service rendu par le système étudié et les consommations indirectes qui y seraient liées !
%Ceci sera développé dans les sections suivantes notamment sur la définition du périmètre.

\subsection{À l'UMET et ailleurs}
Rattachée par l'application à la métallurgie, cette thèse est localisée à l'Unité Matériaux et Transformations, unité mixte de recherche du CNRS 8207 à l'Université de Lille 1.
Plus particulièrement, je suis dans l'équipe MPGM, Métallurgie Physique et Génie des Matériaux.
Le bureau de doctorants et stagiaires était à mon arrivé uniquement composé de simulateurs (formation et altération de l'organisation de la matière au sein des métaux).
Leurs aptitudes au clavier ont contribué aux orientations de ce travail.

C'est au cours de la formation doctorale ``Valoriser ses connaissances avec Wikipédia'' que je découvre l'envers de Wikipédia mais également la Wikiversité.
À cette même période j'ai suivi deux cycles, celui de Wikipédia et celui intitulé ``La propriété intellectuelle au service des doctorants''.
Il n'aura certainement pas échappé aux intéressés (du moins certains et certaines) qu'ils et elles ne sont pas propriétaires des objets en question.
Des rencontres donc dont le lecteur pourra juger de l'influence durant la lecture.

Hors de l'assistance matérielle et de ``la communauté du C6'', mes échanges sur ces travaux eurent lieu avec d'autres laboratoires et chercheurs.
Ma sœur, doctorante au laboratoire \textsc{Psitec} (Lille~3), m'a évidement guidé sur certains aspects en psychologie, mais également sur sa propre expérience de la recherche.
C'est à elle que je dois d'avoir présenté mon travail à la conférence EAWOP2015 à Oslo.
La découverte du web sémantique dans sa pratique n'aurait pu être possible sans l'accord de Christopher Davis sur \textsc{Enipedia} (Delft, Groningen), travaux eux-mêmes identifiés lors d'une discussion durant une école d'été animée par Isabelle \textsc{Blanc} dont les participants comportaient un groupe de \emph{Néerlandais}.
La discussion sur la publication a eu lieu dans le monde académique (notamment avec Joachim \textsc{Schöpfel}, Laboratoire GERiiCO Lille~3), mais les échanges ont également eu lieu avec la société civile et d'autres enseignants chercheurs (syndiqué SNESUP puis Ferc-Sup~CGT, je suis aujourd'hui membre d'ATTAC, de Précaires-ESR, du Parti Pirate, j’interagis avec Catalyst et le mouvement des Communs, Colibris, Nuit-Debout, les wikiversitaires \ldots) en somme tout ceci constitue un terrain hétéroclite mais lié.

\keybox{
Ces quelques lignes, probablement inhabituelles dans ce type de document, rappellent que \textbf{la recherche n'est pas conduite dans un espace clos, vide d'interaction}.
Si \textbf{les conclusions de ce travail sont larges} et modifient le cadre même de l'\gls{ACV}, c'est aussi \textbf{grâce à cette diversité}.
}
