{\large \bfseries
Contribution à la re-conception méthodologique de\\l'évaluation de la soutenabilité par la pensée en cycle de vie, via l'intégration explicite du jugement, le traitement de la multi-dimensionnalité et l'approche alternative libre des données.
%Fondations de l'Évaluation Holistique Opérationnelle\\Re-conception méthodologique de l'évaluation de la soutenabilité par la pensée en cycle de vie et l'artificialisme}
}
~\\

\textbf{Résumé}\\
L'humanité va vers des défis d'ampleurs inégalées dans son histoire.
Pour guider ses décisions sur une voie plus \emph{soutenable} elle a produit divers artefacts.
Toutefois, ils n'ont pas permis de répondre à cette quête de soutenabilité.
Par la re-conception méthodologique de l'Analyse en Cycle de Vie (ACV) via l'artificialisme, découlent les fondations de l'Évaluation Holistique Opérationnelle (ÉHO).

L'intégration explicite du jugement de valeurs, essentielle à des décisions rationnelles, ainsi que l'étude de la multidimensionnalité, en la multifonctionnalité d'abord puis dans sa généralisation, bouleversent les principes établis de l'actuelle pensée en cycle de vie.

Des fondamentaux persistent~:
Le socle commun au modèle canonique de la décision et sa formulation itérative en étapes~: d'intelligence du problème (but), de créativité (inventaire et recherche d'alternatives), de choix (interprétation).

Mais d'autres sont éliminés~:
Le modèle techno-eco-sphère et les notions de flux élémentaires ou produits attenantes, pour ne garder que celles de flux et de système.
La logique des méthodes d'impacts pour ne garder que la description de systèmes technico-sociaux et environnementaux alimentant l'Aide à la Décision MultiCritère.
L'espace des aires de protection fusionnant à celui des besoins (et fonctions attenantes).
Enfin et non des moindres, la logique académique de production et de dissémination des connaissances, dont la rénovation s'accompagne de l'extinction des dominations actuelles en ACV et généralement en Science, au profit d'un web sémantique, interactif, Libre.

Cette thèse, marquant l'union entre l'Évaluation Environnementale et l'Économie, vise donc à passer de l'ACV à l'ÉHO.

~\\
\textbf{Mots clefs~:}\\
Soutenabilité~; Analyse en Cycle de Vie~; Multifonctionnalité~; Multidimensionnalité~; Jugements de valeurs~; Aide à la décision Multicritère~; Conception~; Artificialisme
\newpage
