\section{ADMC, le paradoxe d'\textsc{Arrow}}
\label{Arrow}

Nous faisons appel, au point~\ref{Annexe:Arrow}, au théorème d'impossibilité d'\citeauthor{arrow_difficulty_1950}.

Nous reprenons donc ici l'expression du paradoxe d'\citeauthor{arrow_difficulty_1950}~\cite{arrow_difficulty_1950}

\blockcquote{arrow_difficulty_1950}{
Condition I: The social welfare function is defined for every admissible pair of individual ordering, R1, R2.
Condition I , it should be emphasized, is a restriction on the form of the social welfare function, since we are requiring that for some sufficiently wide range of sets of individual orderings, the social welfare function give rise to a true social ordering.}
\figbox{C'est la condition que nous désignons par `Universalité', ce que nous comprenons comme un ordre total (échelle ordinale pure), sur l'ensemble de la préférence sociale (collective).

C'est une condition que nous rejetons.
Dans nos préconisations (conclusion finale), il s'agit de travailler à une convergence par l’extension du nombre de dimensions de la matrice de jugement.
Mais celle-ci est constamment inconsistante, tant à l'échelle individuelle que collective.
Parmi les solutions pratiques de la détermination des priorités sur la méthodes AHP, nous avons retenu l'extraction de la fraction consistante et le parcours statistique des fréquences de préférences~\footnote{En somme une simulation monté-carlo sur les indéterminations}.
}

\blockcquote{arrow_difficulty_1950}{
Condition 2: If an alternative social state x rises or does not fall in the ordering of each individual without any other change in those orderings and if x was preferred to another alternative y before the change in individual orderings, then x is still preferred to y.}
\figbox{C'est la condition de monotonie de la fonction de préférence.
Plus des individus préfèrent une entité, plus elle est collectivement préférée.}

\blockcquote{arrow_difficulty_1950}{
Condition 3: Let R1, R2, and R'1, R'2 be two sets of individual orderings.
If, for both individuals i and for all z and in a given set of alternatives S, xRiy if and only if xR'iy, then the social choice made from S is the same whether the individual orderings are R1, R2, or R'1, R'2.
(Independence of irrelevant alternatives.)}
\figbox{C'est la condition que nous désignons par `Indépendance des alternatives non-pertinentes', dont l'explication est donnée en corps de thèse.}

\blockcquote{arrow_difficulty_1950}{
Condition 4 : The social welfare function is not to be imposed.}
\figbox{C'est la condition de souveraineté.}

\blockcquote{arrow_difficulty_1950}{
Condition 5: The social welfare function is not to be dictatorial (non-dictatorship).}
\figbox{C'est la condition de non-dictature.}

\blockcquote{arrow_difficulty_1950}{
Possibility Theorem.\\
If there are at least three alternatives among which the members of the society are free to order in any way, then every social function satisfying conditions 2 and 3 and yielding a social ordering satisfying Axioms I and II must be either imposed or dictatorial.
The Possibility Theorem shows that, if no prior assumptions are made about the nature of individual orderings, there is no method of voting which will remove the paradox of voting discussed in Part I, neither plurality voting nor any scheme of proportional representation, no matter how complicated. Similarly, the market mechanism does not create a rational social choice.

[\ldots]

If we exclude the possibility of interpersonal comparison of utility, then the only methods of passing from individual taste to social preferences which will be satisfactory and which will be defined for a wide range of sets of individual orderings are either imposed or dictatorial.}
\figbox{
C'est selon moi la volonté d'unification totale, de consistance totale qui conduit à cette conclusion.
Rien ne nous empêche de rejeter les conditions de jugements imposés ou dictatoriaux et d'accepter, l'incertitude (occurence d'indétermination) i.e. ne plus avoir un ordre total, et donc l'inconsistance au moins partielle sur des fractions de notre structure axiologique.
}