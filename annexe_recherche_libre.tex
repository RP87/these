%\chapter{Recherche Libre}
\section{Plaidoyers du libre et ACV}
\label{sec:Plaidoyers du libre dans l'ACV}

Nous rapportons ici des traductions d'extraits en faveur des outils libres et permettant de prendre la problématique des données pour l'\gls{ACV} à la grandeur de son ampleur, par les données connectées et le web sémantique.
Cette pièce d'annexe est appelé en section~\ref{sec:JSL} mais aussi plus tôt au \ref{subsec:L'informatique de l'ACV}.
Parce que ce thème n'attend personne dans un domaine particulier pour être mis en œuvre, nous reprenons également des extraits pour signifier que cet axe est déjà en marche\ldots ailleurs qu'en ACV pour des données qui lui sont nécessaires.

%\HRule \\[0.3cm]
  
\citeauthor{borkum_health_2014}, discutant de la lisibilité machine des représentation des réglementations GHS/CLP (définition et classification des substances chimiques) conclu ainsi~:
\blockcquote[traduction de la conclusion]{borkum_health_2014}{
Le point central est la disponibilité de l'information chimique précise et digne de confiance.
%Linchpin is availability of accurate and trustworthy chemical information.
}


\blockcquote[traduction]{funtowicz_science_1993}{
Bien que la définition des problèmes ne soit jamais complètement libre de la politique, un débat ouvert garantit que de telles considérations ne sont ni biaisées d'un côté ni secrètes.
%Although the definition of problems is never completely free of politics, an open debate ensures that such considerations are neither one-sided nor covert.
[\ldots]
%When problems lack neat solutions, when environmental and ethical aspects of the issues are prominent, when the phenomena themselves are ambiguous, and when all research techniques are open to methodological criticism, then the debates on quality are not enhanced by the exclusion of all but the specialist researchers and official experts.
%
%
L'extension de la communauté des pairs est alors non seulement un acte éthique ou politique; elle peut enrichir positivement le processus d'investigation scientifique.
%The extension of the peer community is then not merely an ethical or political act; it can positively enrich the processes of scientific investigation.
}

\blockcquote[traduction]{zhang_lca-oriented_2015}{
%(1) Analyse collaborative et modélisation.
L'analyse d'inventaire (ICV) et la modélisation d'un produit complexe est un travail très fastidieux et long.
Il est préférable d'avoir de nombreux analystes dans les différents environnements travaillant en collaboration ensemble à travers le Web pour l'analyse et la modélisation d'ICV.}
[\ldots]

\blockcquote[traduction]{zhang_lca-oriented_2015}{
Un concept du système commun, ouvert, formellement et explicitement défini est le fondement de la compréhension de l'information.
En outre, il est également important d'organiser et de stocker les informations du concept dans un format lisible par l'homme et lisible par machine.
%(1) Collaborative analysis and modeling.
%LCI analysis and modeling of a complex product is a very tedious
%and time-consuming work. It is best to have many analysts in the
%different environment collaboratively work together through Web
%for LCI analysis and modeling.
%
%[\ldots]
%
%An open, shared, and formally and explicitly defined concept
%system is the foundation for information understanding. In addi-
%tion, it is also important to organize and store the concept infor-
%mation in a human-readable and machine-readable format.
}

\blockcquote[traduction de la conclusion sur la problématique p.232]{davis_making_2012}{
Le principal problème que nous avons identifié est que si il est évident que l'utilisation des données dans la recherche doit être plus efficace, il y a encore un grand écart entre nos pratiques actuelles et les avantages qui peuvent être obtenus en mettant à profit la toile.
Le Web est largement rendu possible par des standards ouverts, qui renforcent une toile d'innovations combinatoires aussi dans le domaine de la gestion et l'utilisation des données à travers les communautés de recherche.
%The main problem we identified is that while it is obvious
%that the use of data in research needs to be more efficient, there is still a large gap
%between our current practices and the benefits that may be realized by leveraging
%the Web. The Web is largely enabled by open standards, which reinforce a web
%of combinatorial innovations also in the realm of the management and use of data
%through research communities.
}

\blockcquote[traduction de la conclusion sur les résultats p.234]{davis_making_2012}{
Les mécanismes d’accroissement des capacités et les meilleures pratiques recueillies durant l'analyse sont convertis en exigences de conception fonctionnelle.
Les exigences importantes sont~: permettre de faibles coûts de transaction pour les contributions ; utiliser des standards ouverts, open source et des données ouvertes ; encourager de nombreux yeux ; et la faciliter la découverte et la discussion au sein de la communauté.
%discussions et découverte.
%
%The enabling mechanisms and best practices gathered from the analysis are translated to functional design requirements. Important requirements are: enabling low transaction costs for contributions; using open standards, open source and open data; encouraging many eyes; and facilitating community discussions and discovery.
}


\citetitle{mutel_ideas_2014} (Quelques idées sur une version opensource du logiciel ecoinvent)~\cite{mutel_ideas_2014}.

\citetitle{masanet_reflections_2014}
\blockcquote[traduction, Conclusion]{masanet_reflections_2014}{
Cependant, les MOOCs ne sont pas sans limites structurelles, en particulier celles liées la plupart du temps à du contenu i«verrouillé» 
%However, MOOCs are not without structural limitations, especially related to mostly “locked in” content
}\footnote{Ceci questionne la dénomination de Massive Online \emph{Open} Courses}.

\blockcquote[traduction, p.14]{lamela_footprinted._2011}{
Nous pensons qu'il est nécessaire d'apporter des concepts de données ouvertes à partir du Web à l'information sur les impacts environnementaux (Davis et al, 2010; Zapico et al, 2010).
Cela permettrait d'accroître la transparence, l'ouverture, et de rendre plus facile la création des services pour la soutenabilité en addition aux données.
%We believe that it is necessary to bring open data concepts from the web to environmental
%impact information (Davis et al, 2010; Zapico et al, 2010). This would increase transparency,
%openness, and make it easier to create sustainability services on top of the data.
}

\blockcquote[traduction, p.14]{sayan_contribution_2011}{
Le manque de données standardisées peut être considéré comme un  sous-problème du mouvement des données ouvertes dans la communauté universitaire, dont un nombre important d'universitaires et de scientifiques ont exprimé un désir de données standardisées, disponibles gratuitement en tant que catalyseur pour accélérer le progrès scientifique (Seringhaus, 2007).
%The lack of standardized data can be considered a subset problem of the open data movement in
%the academic community, of which a significant number of academics and scientists have voiced
%a desire for standardized, freely available data as a catalyst for accelerating scientific progress
%(Seringhaus, 2007).
}
\blockcquote[traduction, p.22]{sayan_contribution_2011}{
Étant donné que la collecte de données d'ACV est coûteuse, prend beaucoup de temps, et dépend de l'accès aux bases de données propriétaires, les bases de données ne sont pas toujours libres et ont des restrictions internes sur l'affichage de métadonnées et d'autres détails.
Hisheir et al. (2005) notent que les bases de données existantes ont dû conserver de solides niveaux de confidentialité en masquant les détails du processus parce que les industries (par exemple, dans le cas du secteur chimique) sont réticents à partager trop de détails.
%Since the collection of LCA data is costly, time-intensive, and dependent on access to
%proprietary datasets, databases are not always free and have internal restrictions on viewing
%metadata and other details. Hisheir et al. (2005) note that existing databases have had to retain
%strong levels of privacy by hiding the process details because industries (e.g., in the case the
%chemical industry) are reluctant to share too much detail.
}
\blockcquote[Conclusion, traduction p.104]{sayan_contribution_2011}{
L'ACV est de pertinence réduite lorsque les données et résultat ne sont pas ouvert et utilisable.
%The LCA field is also diminished in its relevance when results and data are not open or useable.
}

\blockcquote[traduction]{belleau_bio2rdf:_2008}{
[\ldots] si l'on veut promouvoir la méthode du web sémantique pour l'intégration de données, \textbf{l'utilisation du logiciel libre open source devrait être encouragée afin d'améliorer la reproductibilité des résultats qui sont publiés dans la littérature.}
Le projet Bio2RDF a été construit à la suite de ces leçons.
%Thirdly, if one wants to promote the semantic web method for data integration, \textbf{the use of free open source software should be encouraged in order to enhance the reproducibility of results that are published in the literature.}
%The Bio2RDF project was built as a result of these lessons.
}
\blockcquote[traduction]{belleau_bio2rdf:_2008}{
Bio2RDF a montré qu'il est possible de changer d'échelle et d'évoluer jusqu'à des millions de documents (dans notre exemple, 163 millions de documents provenant de plus de 20 sources de données différentes).
%Bio2RDF showed that it is possible to scale up to millions of documents (in our example, 163 million documents from more 20 different data sources).
}\footnote{Les bases de données actuelles en \gls{ACV} sont plutôt de l'ordre du millier de set de données.}

%\blockcquote[traduction]{perez_role_2014}{
%Dans cet article, nous analysons l'impact d'une IMS sémantique activée dans un processus d'innovation en matière de soutenabilité.
%En particulier, comment les idées peuvent être enrichies avec des Linked Open Data (LOD)[données ouvertes connectées] contextuelles, plus spécifiquement des données d'\gls{ACV}, afin d'améliorer la compréhension, l'implication et la valeur de l'idée du point de vue de la soutenabilité.
%%In this paper we analyse the impact of a semantic-enabled IMS within a sustainability innovation process. In particular, how ideas can be enriched with contextual Linked Open Data (LOD), especially Life-Cycle Assessment (LCA) data, to improve the understanding, implication and value of the idea from the sustainability perspective.
%}

\blockcquote[traduction]{ingwersen_new_2015}{
Allant de l'avant L'ACV-HT (Harmonization Tool) est destiné à être une composante essentielle de l'architecture de données en ACV (un commun des données) utilisé par les agences fédérales américaines et d'autres fournisseurs de données pour rendre les données représentant les conditions des États-Unis plus accessibles au public.
Il sera également utilisé pour rassembler des données de modèles d'exposition pour la santé humaine avec l'ACV traditionnelle pour évaluer en proximité du terrain le risque pour la santé humaine dans le contexte du cycle de vie pour démontrer les progrès pratiques possibles avec cette nouvelle architecture. L'outil restera open source et disponible gratuitement.
%Moving forward The LCA-HT is intended to be a core component of LCA data architecture (a data commons) used by US federal agencies and other data providers to make data representing US conditions more accessible for public use. It will also be used to bring together data from human health exposure models with traditional LCA for evaluating near-field human health risk in the life cycle context to demonstrate the practical advancements possible with this new architecture. The tool will remain open source and freely available.
}

\blockcquote[traduction]{weidema_bonsai_2014}{
Notre stratégie repose sur 3 principes: [\dots]
(3) Une interface sociale collaborative, en utilisant l'open source et de la technologie innovante des médias sociaux
pour la saisie des données, l'examen et l'interaction avec les communautés d'utilisateurs.
%Our strategy is based on 3 principles: [\ldots]
%(3rd) A collaborative social interface, using open source and innovative social media technology
%for data input, review, and interaction with the user communities.
}

OpenLCA, un outil de modélisation open-source en ACV \citetitle{ciroth_ict_2007}~\cite{ciroth_ict_2007}


%\citetitle{logan_ten_2010}
\blockcquote[traduction]{logan_ten_2010}{
Donc, il y a un besoin croissant de la communauté scientifique à coopérer avec Wikipedia pour assurer que les informations qu'elle contient sont exactes et à jour.
Pour les scientifiques, contribuer à Wikipédia est un excellent moyen de remplir les responsabilités l'engagement public et le partage de l'expertise.
%So there is an increasing need for the scientific community to engage with Wikipedia to ensure that the information it contains is accurate and current. For scientists, contributing to Wikipedia is an excellent way of fulfilling public engagement responsibilities and sharing expertise.
}

\citetitle{madlberger_development_2013}, par \citeauthor{madlberger_development_2013}
\blockcquote{madlberger_development_2013}{
Les technologies pilotées par les données comme les méthodes participatives de détection, les pratiques d'Open Data liés ou des systèmes d'information géographique sont des méthodes utiles pour collecter, agréger et diffuser de l'information.
Cette article décrit la portée du problème, l'approche de la solution et le plan de recherche pour ma thèse de doctorat, qui explore le potentiel de ces technologies pour améliorer la transparence de la soutenabilité des entreprises en utilisant une approche fondée sur le plan scientifique.
%Data-driven technologies like Participatory Sensing methods, Linked Open Data practices or Geographical Information Systems are useful methods to collect, aggregate and disseminate information.
%This paper outlines the problem scope, the solution approach and the research plan for my doctoral thesis which explores the potential of these technologies to improve the transparency of corporate sustainability using a design-science based approach.
}


\citetitle{bonanni_open_2010}, par \citeauthor{bonanni_open_2010}
\blockcquote[traduction]{bonanni_open_2010}{
Nous voulons que l'information sur la soutenabilité soit ouverte, libre et facile à utiliser.
%We want sustainability information to be open, free and easy to use.
%\href{http://opensustainability.info}{http://opensustainability.info}
}
\cite{finnveden_recent_2009}
\blockcquote{finnveden_recent_2009}{
Environmental considerations need to be integrated in many
types of decisions. This includes decisions related to goods and
services. In order to do that, knowledge must be available.
}